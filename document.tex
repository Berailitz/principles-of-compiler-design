\documentclass[UTF8]{ctexart}
    \usepackage{amsmath}
    \usepackage{geometry}
	\usepackage{url}
	\usepackage{verbatim}

    \title{词法分析程序说明文档}
    \author{2016211305班 \ 熊光正 (学号:2016211249)}
    \date{\today}
\begin{document}
\maketitle
\tableofcontents
\section{符号说明}
\begin{enumerate}
	\item $[a-bc-de...]$: 序列中$a$至$b$、$c$至$d$或$e$中的任意字符;
	\item $[\wedge a-bc-de...]$: 不在序列中$a$至$b$、$c$至$d$或$e$中的任意字符;
	\item $languages$: 所有合法字符串;
	\item $ints$: 一个无符号整数;
	\item $octs$: 8进制无符号整数;
	\item $oct = [0-7]$: 0 至 7 的一位阿拉伯数字;
	\item $decs$: 10进制无符号整数;
	\item $dec = [0-9]$: 0 至 9 的一位阿拉伯数字;
	\item $dec1 = [1-9]$: 1 至 9 的一位阿拉伯数字;
	\item $hexs$: 16进制无符号整数;
	\item $hex = [1-9a-eA-E]$: 0 至 9 的一位阿拉伯数字,或$A$至$E$的一位大小写字母;
	\item $floats$: 10进制无符号浮点数;
	\item $float_no_e$: 10进制无符号、没有使用科学记数法表示的浮点数;
	\item $chars$: 字符型;
	\item $char$: 一位任意字符;
	\item $charNoSqBs = [\wedge '\backslash]$: 一位非单引号、也非反斜线(转义符)的字符;
	\item $charInString = [\wedge "]$: 一位非双引号的字符;
	\item $charNoBl = [\wedge \backslash n]$: 一位非换行符的字符,可出现在单行注释中;
	\item $charNoStar = [\wedge *]$: 一位非星号的字符,用于跨行注释的识别;
	\item $charNoBs = [\wedge /]$: 一位非反斜杠的字符,用于跨行注释的识别;
	\item $letter$: 考虑到一位大小写字母字符的功能与下划线相等,我们用$letter$表示一位大小写字母字符或下划线;
	\item $strings$: 字符串型;
	\item $keyword$: 关键字;
	\item $identifier$: 标识符;
\end{enumerate}
注:为便于识别,必要时,我们使用空格分隔正则表达式中的各个符号。
\section{文法的推导}
\subsection{关键字和标识符}
考虑到关键字和标识符均服从正则表达式$letter (letter|ddec|\_)^*$,我们先将其视为同一类型的符号,识别出再进行分类
,其文法为:
$$ identifier \rightarrow letter \ identifier1 $$
$$ identifier1 \rightarrow \varepsilon | letter \ identifier1 | dec \ identifier1 $$
\subsection{整数}
我们分别识别$8$进制、$10$进制和$16$进制的整数
\subsubsection{$8$进制整数}
以$0$开始,由$0$至$7$的整数组成,且长度不小于$2$,即
$$ octs = 0 \ {oct}^+ $$
,其文法为
$$ octs \rightarrow 0 \ octs1 $$
$$ octs1 \rightarrow oct \ octs2 $$
$$ octs2 \rightarrow \varepsilon | oct \ octs2 $$
\subsubsection{$10$进制整数}
若不以科学记数法表示,以$dec1$开始,由$dec$组成;若以科学记数法表示,则在不以科学记数法表示的整数的基础上
附加大小写英语字母$e$,正负号,和任意数字,即加上$0|((E|e)(+|-)?dec+)$。综上所述,其正则式为
$$ decs = 0 | dec1 \ dec+((E|e)(+|-)?dec+)? $$
其文法为
$$ decs \rightarrow dec1 \ decs1  $$
$$ decs1 \rightarrow \varepsilon | dec \ decs1 | E \ decs2 | e \ decs2 $$
$$ decs2 \rightarrow + \ decs3 | - \ decs3 | dec \ decs3  $$
$$ decs3 \rightarrow \varepsilon | dec \ decs3  $$
\subsubsection{$16$进制整数}
以$0x$开始,由$hex$组成,且长度不小于$3$,即
$$ hexs = 0 \ x  \ {hex}^+ $$
其文法为
$$ hexs \rightarrow 0 \ hexs1 $$
$$ hexs1 \rightarrow x \ hexs2 $$
$$ hexs2 \rightarrow hex \ hexs3 $$
$$ hexs3 \rightarrow \varepsilon | hex \ hexs3 $$
\subsubsection{所有整数}
综上所述,整数的文法为
$$ ints \rightarrow 0 \ octs1hexs1dec0 | dec1 \ decs1 $$
$$ octs1hexs1dec0 \rightarrow \varepsilon | oct \ octs2 | x \ hexs2 $$
$$ octs2 \rightarrow \varepsilon | oct \ octs2 $$
$$ hexs2 \rightarrow hex \ hexs3 $$
$$ hexs3 \rightarrow \varepsilon | hex \ hexs3 $$
$$ decs1 \rightarrow \varepsilon | dec \ decs1 | E \ decs2 | e \ decs2 $$
$$ decs2 \rightarrow + \ decs3 | - \ decs3 | dec \ decs3  $$
$$ decs3 \rightarrow \varepsilon | dec \ decs3  $$
依据上述文法识别到整型后,我们通过$atoi$函数将其转换为整数。
\subsection{浮点数}
对于非科学记数法表示的浮点数,其小数点前、后的部分可分别视为一个$10$进制整数;对于科学记数法表示的浮点数,
其指数部分可以视为一个$10$进制有符号或无符号整数。故其正则式为
$$ floats = decs.{dec}^+(e|E)(+|-)dec $$
,其文法为
$$ floats \rightarrow dec1 \ floats1 $$
$$ floats1 \rightarrow dec \ floats1 | . \ floats2 $$
$$ floats2 \rightarrow dec \ floats3 $$
$$ floats3 \rightarrow \varepsilon | dec \ floats3 | e \ floats4 | E \ floats4 $$
$$ floats4 \rightarrow dec \ floats5 $$
$$ floats5 \rightarrow \varepsilon | dec \ floats5 $$
依据上述文法识别到浮点型后,我们通过$atof$函数将其转换为浮点数
\subsection{(单个)字符}
由单引号、非单引号字符、单引号组成,其正则式为
$$ ` \ (charNoSqBs | \backslash \ char) \ ' $$
,其文法为
$$ chars \rightarrow ' \ chars1 $$
$$ chars1 \rightarrow \backslash \ chars2 | charNoSqBs \ chars3 $$
$$ chars2 \rightarrow char \ chars3 $$
$$ chars3 \rightarrow ' $$
\subsection{字符串}
由双引号、非双引号也非换行符的字符、双引号组成,其正则式为
$$ '' \ {(charInString | \backslash \ char)}^* \ '' $$
,其文法为
$$ strings \rightarrow '' \ strings1 $$
$$ strings1 \rightarrow '' | \backslash strings2 | charInString \ strings1 $$
$$ strings2 \rightarrow char \ strings1 $$
\subsection{运算符}
我们借用语法分析中的首符号集的概念,对各个运算符进行分析,得到如下文法
\begin{equation}
	\begin{aligned}
		UnaryOperators \rightarrow & . | ? | \sim | - \ Operators1 | ! \ Operators2 | \% \ Operators3 | \& \ Operators4 \\
		                           & | * \ Operators5 | / | Operators6 | \wedge \ Operators7 | \ | \ Operators8         \\
		                           & | + \ Operators9 | < \ Operators10 | = \ Operators11 | > \ Operators12             \\
		                           & | \colon \ Operators15
	\end{aligned}
\end{equation}
$$ Operators1 \rightarrow \varepsilon | - | = $$
$$ Operators2 \rightarrow \varepsilon | = $$
$$ Operators3 \rightarrow \varepsilon | = $$
$$ Operators4 \rightarrow \varepsilon | \& | = $$
$$ Operators5 \rightarrow \varepsilon | = $$
$$ Operators6 \rightarrow \varepsilon | = $$
$$ Operators7 \rightarrow \varepsilon | = $$
$$ Operators8 \rightarrow \varepsilon | \ | | \ = $$
$$ Operators9 \rightarrow \varepsilon | + | = $$
$$ Operators10 \rightarrow \varepsilon | = | < Operators13 $$
$$ Operators11 \rightarrow \varepsilon | = $$
$$ Operators12 \rightarrow \varepsilon | = | > Operators14 $$
$$ Operators13 \rightarrow \varepsilon | = $$
$$ Operators14 \rightarrow \varepsilon | = $$
$$ Operators15 \rightarrow \varepsilon | \colon $$
\subsection{界符}
分析可知每个界符(终结符)均仅在一个非终结符的首符号集中,故凡遇到终结符界符,即可认定遇到了界符符号,其文法为
$$ delimiters \rightarrow (  \ | \  )  \ | \  [  \ | \  ]  \ | \  \{  \ | \  \}  \ | \  ,  \ | \  ; $$
\subsection{注释}
\subsubsection{单行注释}
由$//$开始,遇到换行符,其正则式为
$$ / \ / \ {charNoBl}^* \ \backslash n $$
,其文法为
$$ commentInLine \rightarrow / \ commentInLine1 $$
$$ commentInLine1 \rightarrow / \ commentInLine2 $$
$$ commentInLine2 \rightarrow \backslash n | charNoBl \ commentInLine2 $$
\subsubsection{跨行注释}
由$/*$开始,遇到$*/$结束,其正则式为
$$ /*({charNoStar}^*|*{charNoBs})^**/ $$
,其文法为
$$ commentCrossLine \rightarrow / \ commentCrossLine1 $$
$$ commentCrossLine1 \rightarrow * \ commentCrossLine2 $$
$$ commentCrossLine2 \rightarrow * \ commentCrossLine3 | charNoStar \ commentCrossLine2 $$
$$ commentCrossLine3 \rightarrow / | charNoBs \ commentCrossLine2 $$
\subsubsection{宏}
为简化程序,我们将以$\#$开始的宏也视为注释处理,其正则式为
$$ \# {charNoBl}^* \ \backslash n $$
,其文法为
$$ commentOfMacros \rightarrow \# \ commentOfMacros1 $$
$$ commentOfMacros1 \rightarrow \backslash n | charNoBl \ commentOfMacros1 $$
\subsubsection{所有注释}
综上所述,注释的文法为
$$ comments \rightarrow / \ comments1 | \# \ commentOfMacros1 $$
$$ comments1 \rightarrow / \ commentInLine2 |* \ commentCrossLine2  $$
$$ commentInLine2 \rightarrow \backslash n | charNoBl \ commentInLine2 $$
$$ commentCrossLine2 \rightarrow * \ commentCrossLine3 | charNoStar \ commentCrossLine2 $$
$$ commentCrossLine3 \rightarrow / | charNoBs \ commentCrossLine2 $$
$$ commentOfMacros1 \rightarrow \backslash n | charNoBl \ commentOfMacros1 $$
\subsection{所有文法}
\begin{equation}
	\begin{aligned}
		languages & \rightarrow 0 \ octs1hexs1dec0 | dec1 \ decs1floats1 | ' \ chars1 | '' \ strings1                                   \\
		          & | - \ Operators1 | ! \ Operators2 | \% \ Operators3  | \& \ Operators4                                              \\
		          & | * \ Operators5  | / \ Operators6comments1 | \wedge \ Operators7                                                   \\
		          & | \ | \ Operators8 | + \ Operators9  | < \ Operators10 | = \ Operators11                                            \\
		          & | > \ Operators12  | \colon \ Operators15 | letter \ identifier1                                                    \\
		          & | \# \ commentOfMacros1  | . | ? | \sim | (  \ | \  )  \ | \  [  \ | \  ]  \ | \  \{  \ | \  \}  \ | \  ,  \ | \  ;
	\end{aligned}
\end{equation}
\begin{equation}
	identifier1 \rightarrow \varepsilon | letter \ identifier1 | dec \ identifier1
\end{equation}
\begin{equation}
	octs1hexs1dec0 \rightarrow \varepsilon | . \ floats2 | oct \ octs2 | x \ hexs2
\end{equation}
\begin{equation}
	octs2 \rightarrow \varepsilon | . \ floats2 | oct \ octs2
\end{equation}
\begin{equation}
	hexs2 \rightarrow hex \ hexs3
\end{equation}
\begin{equation}
	hexs3 \rightarrow \varepsilon | hex \ hexs3
\end{equation}
\begin{equation}
	decs1floats1 \rightarrow \varepsilon | dec \ decs1floats1 | E \ decs2 | e \ decs2 | . \ floats2
\end{equation}
\begin{equation}
	decs2 \rightarrow + \ decs3 | - \ decs3 | dec \ decs3
\end{equation}
\begin{equation}
	decs3 \rightarrow \varepsilon | dec \ decs3
\end{equation}
\begin{equation}
	floats2 \rightarrow dec \ floats3
\end{equation}
\begin{equation}
	floats3 \rightarrow \varepsilon | dec \ floats3 | e \ floats4 | E \ floats4
\end{equation}
\begin{equation}
	floats4 \rightarrow dec \ floats5
\end{equation}
\begin{equation}
	floats5 \rightarrow \varepsilon | dec \ floats5
\end{equation}
\begin{equation}
	chars1 \rightarrow \backslash \ chars2 | charNoSqBs \ chars3
\end{equation}
\begin{equation}
	chars2 \rightarrow char \ chars3
\end{equation}
\begin{equation}
	chars3 \rightarrow '
\end{equation}
\begin{equation}
	strings1 \rightarrow '' | \backslash strings2 | charInString \ strings1
\end{equation}
\begin{equation}
	strings2 \rightarrow char \ strings1
\end{equation}
\begin{equation}
	\begin{aligned}
		Operators6comments1 \rightarrow \varepsilon | = & | / \ commentInLine2   \\
		                                                & |* \ commentCrossLine2
	\end{aligned}
\end{equation}
\begin{equation}
	commentInLine2 \rightarrow \backslash n | charNoBl \ commentInLine2
\end{equation}
\begin{equation}
	\begin{aligned}
		commentCrossLine2 \rightarrow & * \ commentCrossLine3            \\
		                              & | charNoStar \ commentCrossLine2
	\end{aligned}
\end{equation}
\begin{equation}
	commentCrossLine3 \rightarrow / | charNoBs \ commentCrossLine2
\end{equation}
\begin{equation}
	commentOfMacros1 \rightarrow \backslash n | charNoBl \ commentOfMacros1
\end{equation}
\begin{equation}
	Operators1 \rightarrow \varepsilon | - | =
\end{equation}
\begin{equation}
	Operators2 \rightarrow \varepsilon | =
\end{equation}
\begin{equation}
	Operators3 \rightarrow \varepsilon | =
\end{equation}
\begin{equation}
	Operators4 \rightarrow \varepsilon | \& | =
\end{equation}
\begin{equation}
	Operators5 \rightarrow \varepsilon | =
\end{equation}
\begin{equation}
	Operators7 \rightarrow \varepsilon | =
\end{equation}
\begin{equation}
	Operators8 \rightarrow \varepsilon | \ | | \ =
\end{equation}
\begin{equation}
	Operators9 \rightarrow \varepsilon | + | =
\end{equation}
\begin{equation}
	Operators10 \rightarrow \varepsilon | = | < Operators13
\end{equation}
\begin{equation}
	Operators11 \rightarrow \varepsilon | =
\end{equation}
\begin{equation}
	Operators12 \rightarrow \varepsilon | = | > Operators14
\end{equation}
\begin{equation}
	Operators13 \rightarrow \varepsilon | =
\end{equation}
\begin{equation}
	Operators14 \rightarrow \varepsilon | =
\end{equation}
\begin{equation}
	Operators15 \rightarrow \varepsilon | \colon
\end{equation}
\section{程序功能}
\subsection{识别合法符号及注释}
依据上述文本,本程序可以识别科学记数法或非科学记数法表示的$8$、$10$、$16$进制的无符号整数和浮点数、带转义或不带转义的单个字符或字符串
,各类括号或分号等界符,以及文法中列举的各类操作符。词法分析程序会识别并跳过行内和跨行注释,和$\#$开始的宏指令。
\subsection{统计文本信息}
本程序可以统计并输出源文件的行数和字符总数。
\subsection{错误处理}
本程序可以识别源文件中出现的词法错误,报告其所在行数、列数,并继续分析后面的源代码。对于字符型中缺失后单引号,词法处理程序可以自动纠正。
\section{运行流程}
根据需求分批读取字符至缓冲区,依次分析并形成符号串,对标识符建立符号表,同时统计行数、字符串、符号数和错误数。
若出现自动机无法识别的字符,尝试进行纠正并继续运行至文件尾。
\section{使用说明}
运行程序后根据提示输入目标文件名,或将目标文件拖拽至词法分析程序的可执行文件上。
\section{样例使用说明}
\subsection{合法源文件}
合法源文件示例
\begin{verbatim}
#include <iostream>
#include <string>

/*
    合法源文件。
*/

int main(int argc, char const *argv[])
{
    const int a = ~0, b=11;
    double d = 0.3e2 + i;
    int j = 0xa1, k = 011;
    char c = '\'';
    std::string s = "string value\n";
    std::cout << s;
    return 0;
}
\end{verbatim}
将$legal.cpp$文件拖拽至$main.exe$文件上,输出
\begin{verbatim}
Line 1
<EmptyToken, 1, `#include <iostream>`>

Line 2:
<EmptyToken, 1, `#include <string>`>

Line 3:

Line 4:

Line 5:

Line 6:
<EmptyToken, 1, `/*
    合法源文件。
*/`>

Line 7:

Line 8:
<KeywordToken, 40, `int`>
<IdentifierToken, 0, `main`>
<DelimiterToken, 0, `(`>
<KeywordToken, 40, `int`>
<IdentifierToken, 1, `argc`>
<DelimiterToken, 6, `,`>
<KeywordToken, 12, `char`>
<KeywordToken, 18, `const`>
<BinaryOperatorToken, 9, `*`>
<IdentifierToken, 2, `argv`>
<DelimiterToken, 2, `[`>
<DelimiterToken, 3, `]`>
<DelimiterToken, 1, `)`>

Line 9:
<DelimiterToken, 4, `{`>

Line 10:
<KeywordToken, 18, `const`>
<KeywordToken, 40, `int`>
<IdentifierToken, 3, `a`>
<BinaryOperatorToken, 14, `=`>
<UnaryOperatorToken, 3, `~`>
<IntToken, 0, `0`>
<DelimiterToken, 6, `,`>
<IdentifierToken, 4, `b`>
<BinaryOperatorToken, 14, `=`>
<IntToken, 11, `11`>
<DelimiterToken, 7, `;`>

Line 11:
<KeywordToken, 26, `double`>
<IdentifierToken, 5, `d`>
<BinaryOperatorToken, 14, `=`>
<FloatToken, 30.000000, `0.3e2`>
<BinaryOperatorToken, 7, `+`>
<IdentifierToken, 6, `i`>
<DelimiterToken, 7, `;`>

Line 12:
<KeywordToken, 40, `int`>
<IdentifierToken, 7, `j`>
<BinaryOperatorToken, 14, `=`>
<IntToken, 0, `0xa1`>
<DelimiterToken, 6, `,`>
<IdentifierToken, 8, `k`>
<BinaryOperatorToken, 14, `=`>
<IntToken, 11, `011`>
<DelimiterToken, 7, `;`>

Line 13:
<KeywordToken, 12, `char`>
<IdentifierToken, 9, `c`>
<BinaryOperatorToken, 14, `=`>
<CharToken, \, `'\''`>
<DelimiterToken, 7, `;`>

Line 14:
<IdentifierToken, 10, `std`>
<BinaryOperatorToken, 12, `::`>
<IdentifierToken, 11, `string`>
<IdentifierToken, 12, `s`>
<BinaryOperatorToken, 14, `=`>
<StringToken, `"string value\n"`>
<DelimiterToken, 7, `;`>

Line 15:
<IdentifierToken, 10, `std`>
<BinaryOperatorToken, 12, `::`>
<IdentifierToken, 13, `cout`>
<BinaryOperatorToken, 5, `<<`>
<IdentifierToken, 12, `s`>
<DelimiterToken, 7, `;`>

Line 16:
<KeywordToken, 58, `return`>
<IntToken, 0, `0`>
<DelimiterToken, 7, `;`>

Line 17:
<DelimiterToken, 5, `}`>

Line 18:
Lexer finished.

Token counters:
EmptyToken: 3
IdentifierToken: 26
KeywordToken: 0
IntToken: 5
FloatToken: 1
CharToken: 1
StringToken: 1
UnaryOperatorToken: 1
BinaryOperatorToken: 12
TernaryOperatorToken: 0
DelimiterToken: 16
18 lines.
262 characters.
0 errors.

Press Enter to exit.
\end{verbatim}
词法分析程序分析出了所有合法的符号,建立了标识符表,并得到了行数、符号数与错误数。
\subsection{含错误源文件}
含错误源文件示例
\begin{verbatim}
#include <iostream>
#include <string>

/*
    含错误源文件。
*/

int main(int argc, char const *argv[])
{
    const int i = ~0., b = 11;
    double d = 0.3e2 + i;
    int j = 0xz1, k = 011;
    char c = '\';
    std::string s = "string value\n";
    std::cout << s;
    return 0;
}
\end{verbatim}
将$illegal.cpp$文件拖拽至$main.exe$文件上,输出
\begin{verbatim}
Line 1
<EmptyToken, 1, `#include <iostream>`>

Line 2:
<EmptyToken, 1, `#include <string>`>

Line 3:

Line 4:

Line 5:

Line 6:
<EmptyToken, 1, `/*
    含错误源文件。
*/`>

Line 7:

Line 8:
<KeywordToken, 40, `int`>
<IdentifierToken, 0, `main`>
<DelimiterToken, 0, `(`>
<KeywordToken, 40, `int`>
<IdentifierToken, 1, `argc`>
<DelimiterToken, 6, `,`>
<KeywordToken, 12, `char`>
<KeywordToken, 18, `const`>
<BinaryOperatorToken, 9, `*`>
<IdentifierToken, 2, `argv`>
<DelimiterToken, 2, `[`>
<DelimiterToken, 3, `]`>
<DelimiterToken, 1, `)`>

Line 9:
<DelimiterToken, 4, `{`>

Line 10:
<KeywordToken, 18, `const`>
<KeywordToken, 40, `int`>
<IdentifierToken, 3, `i`>
<BinaryOperatorToken, 14, `=`>
<UnaryOperatorToken, 3, `~`>
<ERROR at (10, 22): a digit excepted, maybe `0`? Jump to the next line.>
<FloatToken, 0.000000, `0., `>

Line 11:
<KeywordToken, 26, `double`>
<IdentifierToken, 4, `d`>
<BinaryOperatorToken, 14, `=`>
<FloatToken, 0.000000, `0.3e2`>
<BinaryOperatorToken, 7, `+`>
<IdentifierToken, 3, `i`>
<DelimiterToken, 7, `;`>

Line 12:
<KeywordToken, 40, `int`>
<IdentifierToken, 5, `j`>
<BinaryOperatorToken, 14, `=`>
<ERROR at (12, 15): HEX word excepted, maybe `0`? Jump to the next line.>
<IntToken, 0, `0xz1`>

Line 13:
<KeywordToken, 12, `char`>
<IdentifierToken, 6, `c`>
<BinaryOperatorToken, 14, `=`>
<ERROR at (13, 17): missing `'` ? Jump to the next line.>
<CharToken, \, `'\';
`>

Line 14:

Line 15:
<IdentifierToken, 7, `std`>
<BinaryOperatorToken, 12, `::`>
<IdentifierToken, 8, `cout`>
<BinaryOperatorToken, 5, `<<`>
<IdentifierToken, 9, `s`>
<DelimiterToken, 7, `;`>

Line 16:
<KeywordToken, 58, `return`>
<IntToken, 0, `0`>
<DelimiterToken, 7, `;`>

Line 17:
<DelimiterToken, 5, `}`>

Line 18:
Lexer finished.

Token counters:
EmptyToken: 3
IdentifierToken: 21
KeywordToken: 0
IntToken: 2
FloatToken: 2
CharToken: 1
StringToken: 0
UnaryOperatorToken: 1
BinaryOperatorToken: 8
TernaryOperatorToken: 0
DelimiterToken: 10
18 lines.
266 characters.
3 errors.

Press Enter to exit.
  \end{verbatim}
词法分析程序分析出了所有合法的符号,并找到了第$10$、$12$和$13$行的错误,尝试修正并继续运行至文件末尾,输出错误数$3$。
\section{完成过程中遇到的问题}
\subsection{程序在$g++$下出现的问题}
若使用$g++$(版本号:$8.1.0$)编译本程序,程序有可能重复分析部分代码,但若使用$clang++$(版本号:$7.0.0$)或$Visual Studio 2017$编译
,则不会出现重复分析的问题。此类情况是因为读取源文件使用的$ifstream::read$函数在不同的编译器下行为不同。
\subsection{分析含中文字符的源代码}
在中文$Windows \ 10$环境中,控制台默认使用$GB2312$作为默认编码,若源文件不为$GB2312$编码,则词法分析程序的输出可能会出现乱码。
\section{备注}
\begin{enumerate}
	\item 词法来源:编译器$Clang++$上进行的测试(版本号:$3.9.1$)和互联网($cppreference.com$等);
	\item 假设字符均合法;
	\item 将所有$($, $)$, $[$, $]$, $\{$, $\}$, $,$视为界符,即
	      \begin{enumerate}
		      \item 将$[index]$中的括号视为界符,数字视为常数、标识符或表达式
		      \item 将$(type)$中的括号视为界符,类型视为关键字或标识符
		      \item 将${value1, value2}$中的$,$视为界符
	      \end{enumerate}
	\item 将$static\_cast$, $dynamic\_cast$, $const\_cast$, $reinterpret\_cast$, $new$, $delete$, $typeid$, $noexcept$, $alignof$视为关键字
	\item 暂不考虑运算符$sizeof$, $sizeof...$
	\item 将$=$, $-=$, $+=$, $*=$, $/=$, $\%=$, $\wedge=$, $\&=$, $|=$, $<<$, $>>$, $>>=$, $<<=$, $?$, $:$视为运算符
	\item 暂不考虑``$,$''作为运算符时,先后计算其左值和右值,并返回右值的特性
	\item $IntTokenType$, $FloatTokenType$均为无符号数,故将$(+|-)(digit|float)$中的$+$, $-$视为运算符
\end{enumerate}
\end{document}
