%!TEX program = xelatex
\documentclass[UTF8]{ctexart}
    \usepackage{amsmath}
    \usepackage{geometry}
	\usepackage{graphicx}
	\usepackage[utf8]{inputenc}
	\usepackage{listings}
	\usepackage{url}
	\usepackage{verbatim}
	
	\graphicspath{ {./images/} }

    \title{$LL(1)$语法词法分析程序说明文档}
    \author{2016211305班 \ 熊光正 (学号:2016211249)}
    \date{\today}
\begin{document}
\lstset{numbers=left,frame=single}
\maketitle
\tableofcontents
\clearpage
\section{概述}
本程序从用户输入或指定的文本文件读取文法四元组,依据算法$4.2$等检查、改造文法,消除空产生式,消除左递归,提前左公因子,
构建$FIRST$集、$FOLLOW$集和预测分析表,依据算法$4.1$等分析输入的句子,并在必要时进行错误处理。
\section{数据结构}
程序使用的数据结构如下:
\begin{table}[!h]
    \centering
    \caption{数据结构表}
    \begin{tabular}{|c|c|c|}
    \hline
    逻辑结构 & 数据结构 & 备注 \\ \hline
    终结符  &  $Terminal$ & 即$string$  \\ \hline
    非终结符  & $Nonterminal$  &   \\ \hline
    文法规则 & $Rule$  &   \\ \hline
    非终结符表 & $unordered\_map<symbol, Nonterminal>$  &   \\ \hline
    候选式 & $vector<symbol>$  & $symbol$泛指所有符号  \\ \hline
    分析表中的行 & $unordered\_map<Terminal, int>$  & 整数表示对应候选式的序号  \\ \hline
    符号栈 & $vector<Terminal>$  &   \\ \hline
    \end{tabular}
    \end{table}
\section{程序功能}
\subsection{识别文法的形式定义}
程序从字符串识别文法,将其转换成内部表示的非终结符表、终结符表、开始符号和产生式列表。
\subsection{检查并改造文法}
程序检查识别的文法是不是$LL(1)$文法,若不是,则尝试改造文法,即消除空产生式、消除直接和间接左递归、提取公共左因子。
\subsection{构建预测分析表}
程序依次构建$FIRST$集和$FOLLOW$集,并依次建立带同步标记的预测分析表。
\subsection{分析输入的句子}
程序依据上述预测分析表,分析输入的句子,若检测到错误,进行错误处理。
\section{程序运行流程及算法}
\subsection{语法构建}
\subsubsection{输入文法形式定义的词法、语法分析}
程序使用换行符分割输入流中的非终结符、终结符、开始符号和各产生式等参数。对于各参数,程序使用空格分隔其中的各个符号,形成单词符号串。
程序依据文法形式定义的语法处理上述单词符号串,并构建由左部非终结符和右部候选式列表组成的$Rule$语法对象。
程序遍历上述语法对象中的候选式,将其加入内部表示的语法结构。
程序不对此处的分析流程进行错误处理。
\subsubsection{消除空产生式}
\subsubsection{消除直接和间接左递归}
\subsubsection{提取左公因子}
\subsubsection{计算$FIRST$集}
\subsubsection{计算$FOLLOW$集}
\subsubsection{构建带同步记号的预测分析表}
程序基于算法$4.2$构建预测分析表。程序遍历各个非终结符,遍历其候选式列表,计算各个候选式$candidate$的$FIRST$集。
对于其中除空产生式外的元素$word$,将偶对$(word, candidate)$插入分析表;
对于空产生式,则遍历该非终结符的$FOLLOW$集中的元素$follower$,将偶对$(follower, \varepsilon)$插入分析表。
若在上述插入过程中发现对应位置已有元素,则该文法不是$LL(1)$文法,抛出错误。
若未发现错误,则遍历非终结符的$FOLLOW$集中的元素$follower$,若表中对应位置上没有记录,则将偶对$(follower, synch)$插入分析表。
\subsection{语法分析}
程序使用上述预测分析表分析输入的句子,输出分析过程和结果。
\section{输入输出说明}
程序可以交互式地从控制台接受用户输入,也可在运行时指定文件名,从文件自动读入,并输出至控制台。
\subsection{交互式运行}
\subsection{从文件读取输入}
本程序支持在$64$位中文$Windows \ 10$环境下运行,没有其他运行时依赖。
\subsection{输入要求}
\section{开发环境}
\section{运行环境}
\section{测试样例说明}
\subsection{合法源文件样例}
\subsubsection{基础测试样例:$legal\_1.cpp$}
本样例测试标识符、科学记数法或非科学记数法表示的$10$进制整数、浮点数、基本运算符和界符的识别及注释的处理等基本要求,源代码中不含词法错误。
\section{备注}
\begin{enumerate}
	\item 词法来源:编译器$Clang++$上进行的测试(版本号:$3.9.1$)和互联网($cppreference.com$等);
	\item 假设字符均合法;
	\item 将所有$($, $)$, $[$, $]$, $\{$, $\}$, $,$视为界符,即
	      \begin{enumerate}
		      \item 将$[index]$中的括号视为界符,数字视为常数、标识符或表达式
		      \item 将$(type)$中的括号视为界符,类型视为关键字或标识符
		      \item 将${value1, value2}$中的$,$视为界符
	      \end{enumerate}
\end{enumerate}
\end{document}
