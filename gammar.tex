\documentclass{article}
    \usepackage{amsmath}
    \usepackage{amssymb}
    \usepackage{breqn}
    \usepackage{geometry}

    \geometry{a4paper,scale=0.8}
\begin{document}
\begin{equation}
	\begin{aligned}
		languages & \rightarrow 0 \ octs1hexs1 | dec1 \ decs1floats1 | ' \ chars1 | '' \ string1               \\
		          & | - \ UnaryOperators1 | ! \ UnaryOperators2 | \% \ UnaryOperators3                         \\
		          & | \& \ UnaryOperators4 | * \ UnaryOperators5                                               \\
		          & | / \ UnaryOperators6comments1 | \wedge \ UnaryOperators7                                  \\
		          & | \ | \ UnaryOperators8 | + \ UnaryOperators9                                              \\
		          & | < \ UnaryOperators10 | = \ UnaryOperators11                                              \\
		          & | > \ UnaryOperators12 | letter \ identifier1                                              \\
		          & | : | ? | \sim | (  \ | \  )  \ | \  [  \ | \  ]  \ | \  \{  \ | \  \}  \ | \  ,  \ | \  ;
	\end{aligned}
\end{equation}
\begin{equation}
	identifier1 \rightarrow \varepsilon | letter \ identifier1 | dec \ identifier1
\end{equation}
\begin{equation}
	octs1hexs1 \rightarrow oct \ octs2 | x \ hexs2
\end{equation}
\begin{equation}
	octs2 \rightarrow \varepsilon | oct \ octs2
\end{equation}
\begin{equation}
	hexs2 \rightarrow hex \ hexs3
\end{equation}
\begin{equation}
	hexs3 \rightarrow \varepsilon | hex \ hexs3
\end{equation}
\begin{equation}
	decs1floats1 \rightarrow \varepsilon | dec \ decs1floats1 | E \ decs2 | e \ decs2 | . \ floats2
\end{equation}
\begin{equation}
	decs2 \rightarrow + \ decs3 | - \ decs3 | dec \ decs3
\end{equation}
\begin{equation}
	decs3 \rightarrow \varepsilon | dec \ decs3
\end{equation}
\begin{equation}
	floats2 \rightarrow dec \ floats3
\end{equation}
\begin{equation}
	floats3 \rightarrow \varepsilon | dec \ floats3 | e \ floats4 | E \ floats4
\end{equation}
\begin{equation}
	floats4 \rightarrow dec \ floats5
\end{equation}
\begin{equation}
	floats5 \rightarrow \varepsilon | dec \ floats5
\end{equation}
\begin{equation}
	chars1 \rightarrow charNoSq \ chars2
\end{equation}
\begin{equation}
	chars2 \rightarrow '
\end{equation}
\begin{equation}
	string1 \rightarrow '' | charInString \ string1
\end{equation}
\begin{equation}
	\begin{aligned}
		UnaryOperators6comments1 \rightarrow \varepsilon | = & | / \ commentInLine2   \\
		                                                     & |* \ commentCrossLine2
	\end{aligned}
\end{equation}
\begin{equation}
	commentInLine2 \rightarrow \backslash n | charNoBl \ commentInLine2
\end{equation}
\begin{equation}
	\begin{aligned}
		commentCrossLine2 \rightarrow & * \ commentCrossLine3            \\
		                              & | charNoStar \ commentCrossLine2
	\end{aligned}
\end{equation}
\begin{equation}
	commentCrossLine3 \rightarrow / | charNoBs \ commentCrossLine2
\end{equation}
\begin{equation}
	UnaryOperators1 \rightarrow \varepsilon | - | =
\end{equation}
\begin{equation}
	UnaryOperators2 \rightarrow \varepsilon | =
\end{equation}
\begin{equation}
	UnaryOperators3 \rightarrow \varepsilon | =
\end{equation}
\begin{equation}
	UnaryOperators4 \rightarrow \varepsilon | \& | =
\end{equation}
\begin{equation}
	UnaryOperators5 \rightarrow \varepsilon | =
\end{equation}
\begin{equation}
	UnaryOperators7 \rightarrow \varepsilon | =
\end{equation}
\begin{equation}
	UnaryOperators8 \rightarrow \varepsilon | \ | | \ =
\end{equation}
\begin{equation}
	UnaryOperators9 \rightarrow \varepsilon | + | =
\end{equation}
\begin{equation}
	UnaryOperators10 \rightarrow \varepsilon | = | < UnaryOperators13
\end{equation}
\begin{equation}
	UnaryOperators11 \rightarrow \varepsilon | =
\end{equation}
\begin{equation}
	UnaryOperators12 \rightarrow \varepsilon | = | > UnaryOperators14
\end{equation}
\begin{equation}
	UnaryOperators13 \rightarrow \varepsilon | =
\end{equation}
\begin{equation}
	UnaryOperators14 \rightarrow \varepsilon | =
\end{equation}

\end{document}
